\section*{Resumen}

Hoy en día, es indudable la utilización de un computador para realizar análisis más profundos, dado que de otro modo el cálculo se vuelve complejo, por esto es fundamental contar con algoritmos y/o software que nos permitan mejorar las respuestas de tiempo al realizar dichos análisis.
\\
En el presente trabajo se desarrolla un software, el cual tiene el objetivo de ser utilizado por usuarios que no cuenten con conocimientos en el área de programación, por tanto no puedan programar el algoritmo que resuelve o procesa consultas kNN ni secuencial ni paralelamente. Este software está desarrollado de manera que pueda ser intuitivo en su utilización, donde solo se deben cargar los datos, tanto la base de datos como las consultas. Éstas deben estar en un archivo de texto plano y para esto posee botones de examinar y localizar el archivo en cuestión. No obstante, puede ser que el usuario no sepa bien el nombre de archivo, o no seleccione el archivo correcto (tanto la base de datos como la consulta), de manera que el software cuente con una vista previa de dichos archivos, de tal manera que el usuario pueda verificar los datos antes de realizar algún procedimiento.\\

El software cuenta con una versión del algoritmo secuencial kNN, algoritmo paralelo multi-hilo kNN (utilizando la librería OpenMP), algoritmo paralelo sobre un coprocesador Intel Xeon Phi (utilizando la librería OpenMP), algoritmo paralelo en GPU (utilizando la extensión de C denominada CUDA).\\
Así como el algoritmo secuencial utiliza como parámetros de entrada la base de datos y las consultas, también el usuario debe indicar la dimensión de los objetos de la base de datos y la cantidad de elementos más cercanos deseados (K).\\

Cabe mencionar que los resultados obtenidos mediante nuestro software, pueden ser exportados a los siguientes formatos XLS (archivos de Excel), DOC (archivos word), TXT (archivos de texto plano), PDF (formato de documento portátil), de forma que el usuario pueda elegir el formato que más le acomode. Para aquellos que si poseen conocimientos de programación, pueden añadir sus propios métodos utilizando las entradas establecidas para cada caso.\\ 