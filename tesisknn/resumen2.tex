\section*{Resumen}
Hoy en día, es indudable la utilización de un computador para realizar análisis más profundos, dado que de otro modo se emplearía demasiado tiempo, por esto es fundamental contar con algoritmos y/o software que nos permitan mejorar las respuestas de tiempo al realizar dichos análisis.
\\\\
En el presente trabajo se desarrolla un software el cual fue pensado para ser utilizado por usuarios que no cuenten con conocimientos en el área de programación, por tanto no puedan programar el algoritmo Knn ni secuencial ni paralelamente. Este software esta desarrollado de manera que pueda ser intuitiva su utilización, donde solo deben cargar los datos, tanto la base de datos como las consultas, estas deben estar en un archivo de texto plano, para estos posee botones para examinar y localizar en archivo en cuestión, no obstante puede ser que el usuario no sepa bien el nombre de archivo o no seleccione el archivo correcto (tanto base de datos como consulta), de manera que el software cuenta con una vista previa de dichos archivos de tal manera el usuario puede verificar los datos antes de realizar algún procedimiento.\\\\

No obstante el proceso anteriormente descrito es válido para todos los algoritmos K-nn que posee el software, mas no es similar para todos los algoritmos debido a sus entradas particulares, dado que el software cuenta con el algoritmo secuencial Knn, algoritmo paralelo milti-hilos knn (utilizando la librería OpenMP), algoritmo paralelo Intel Xeon Phi (Utilizando la librería OpenMP), algoritmo paralelo GPU (utilizando la librería de paralelismo en CUDA).\\
De manera que el algoritmo secuencial utiliza como campos de entrada la base de datos y la consultas, además el usuario debe indicar el tamaño del vector, como también la cantidad de elementos más cercanos deseados (K). A su vez los algoritmos paralelos (Multi-Hilos y Intel Xeon Phi) utilizan las bases de datos y las consultas, además el usuario debe indicar el tamaño del vector, como también la cantidad de elementos más cercanos deseados(K) de la misma manera que lo hace el algoritmo anterior pero estos algoritmo cuentan con una diferencia debido a que se debe indicar el número de hilos con los que se desea que se realice el proceso. Por defecto el software reconoce la cantidad de procesadores que tiene el computador y los establece como la cantidad de hilos a utilizar, este dato es solo un dato sugerido de manera que el usuario puede poner el numero que el desee. A su vez para la GPU se estableció un valor por defecto debido a las características de la implementación del algoritmo en GPU.\\\\

Cabe mencionar que los resultados pueden ser exportados a los siguientes formatos XLS (Archivos de Excel), DOC (Archivos word), TXT (Archivos de texto plano), PDF (Formato de documento portátil), de forma que el usuario pueda elegir el formato que más le acomode. A su vez para aquellos que si poseen conocimientos de programación pueden añadir sus métodos utilizando las entradas establecidas para cada caso. Dado que el software cuenta con un un menú para añadir nuevos métodos del algoritmo, de modo que el usuario en cuestión debe seleccionar a que tipo de algoritmo corresponde (ya sea secuencial, paralelo milti-hilos, paralelo Xeon phi, paralelo GPU). con esto debe seleccionar su archivo fuente (respetando ciertas características establecidas previamente), de manera similar al caso de las bases de datos y consultas el software cuenta con una vista previa para el chequeo correcto de el archivo cargado, además el usuario debe indicar el nombre del menú que desee.\\\\  