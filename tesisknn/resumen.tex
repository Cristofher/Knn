\section*{Resumen}

Un método de emparejamiento de huellas digitales, ejecutado y adaptado en plataformas paralelas se presentan en ésta tésis. Durante un largo periodo estas Herramientas han sido eficaces para el reconocimiento Humano, debido a la singularidad, universalidad e invariabilidad de las huellas dactilares. En la Literatura existen varios métodos para el reconocimiento de Huellas y entre ellas están las minucias, quienes  destacan como las técnicas más relevantes debido a su carácter discriminatorio. A causa de los buenos resultados, se procede en poner a prueba un algoritmo basado en Minucias $"\; Minutia\;Cylinder\;Code\;"$ $(MCC)$. Quien tiene capacidad de proporcionar resultados precisos en cuanto a la igualdad entre dos Huellas digitales. Sin embargo, Cuando se operan grandes Bases de datos, la Identificación de Huellas digitales puede ser una tarea ineficaz, debido a los altos tiempos de computación de los algoritmos de coincidencia de huellas digitales. Atacando a este problema, y en la necesidad de acortar los tiempos de coincidencia, el algoritmo MCC es adaptado para trabajar en las plataformas paralelas, tanto en $OpenMp$ como en $GPU$, acelerando los tiempos de respuesta, sobre el parecido de una Huella con respecto a otra. Resultados con otros algoritmos y plataformas permiten establecer una comparación objetiva respecto del desempeño de otros métodos del estado del arte. Finalmente, resultados sobre la base de datos usada muestran el alcance de las plataformas propuestas.



%En este trabajo, se utiliza un método de identificación de huellas digitales. El método presentado, se acelerará en plataformas paralelas, tales como, GPU y OpenMP. 