\chapter[Introducción]{\label{ch:intro}Introducción}

En esta tesis se propone un software pensado en usuarios que poseen poco o nulo conocimientos de programación, de modo que puedan utilizar este software para resolver consultas kNN. Para esto, se han seleccionado diferentes versiones de algoritmos que resuelven consultas kNN sobre las siguientes plataformas: CPU, multi-núcleo, Intel Xeon Phi [ref], GPU [ref] propuestas en la lieratura. En este software, se presentan cuatro algoritmos diferentes: algoritmo secuencial, algoritmo multi-hilo, algoritmo sobre una plataforma Intel Xeon Phi, algoritmo sobre una plataforma GPU.
Se utilizo JAVA 8 \cite{java} para programar la interfaz gráfica del software debido a que este lenguaje es soportado en los distintos sistemas operativos: Windows en sus versiones 10, 8.x (Escritorio), 7 SP1, Vista SP2, Windows Server 2008 R2 SP1 (64 bits), Windows Server 2012 y 2012 R2 (64 bits) y los recursos necesarios son 128 MB de RAM, un espacio minino de 124 MB, procesador pentium 2 a 266 MHz. También en MAC OSx equipos Mac con un procesador Intel que ejecuta Mac OS X 10.8.3+, 10.9+, donde la instalación se debe realizar en modo administrador. En linux en cambio está disponible para Oracle Linux desde la versión 5.5+ o en versiones superiores, en Red Hat Enterprise Linux desde la versión 5.5+ o en versiones superiores, Suse Linux Enterprise Server desde 10 SP2+ o versiones superiores y en Ubuntu Linux en desde versiones 12.04 LTS o en versiones superiores.\\

Para la programación de los algoritmos se empleó el lenguaje de programación C \cite{c}, dado que este lenguaje puede ser utilizado en diversos sistemas operativos, además de poder utilizar librerías que permiten el uso del paralelismo en distintos tipos de plataformas paralelas, como son multi-núcleo, Intel Xeon Phi y GPU.\\
\\
Las plataformas paralelas que se emplean en esta tesis son las mencionadas anteriormente, tanto para multi-núcleo e Intel Xeon Phi se utiliza la librería OpenMP \cite{libroOpenMP}\cite{libroPthreads}. Esta librería está pensada para los lenguajes C, C++ y Fortran, disponible además para diversos tipos de arquitecturas, incluidas algunas plataformas como Windows y Unix.
A su vez para GPU se cuenta con librerías propuestas por NVIDIA y CUDA \cite{cuda} la cual permite trabajar las concurrencias en GPU, poder utilizar los hilos y multiprocesadores que propone NVIDIA.   


kNN \textit{(K nearest neighbors)} es un tipo de consulta de minería de datos más utilizado, este es un clasificador basado en instancias (aprendizaje por analogía). Se buscan los casos más parecidos y la clasificación se realiza en función de la clase analizada, siendo conocido por ser un paradigma perezoso debido a ser un análisis exhaustivo con la consulta y todos lo elementos del conjunto.

Si bien el algoritmo kNN implica alto cómputo, al ser exhaustivo (evaluar cada elemento del conjunto) es exacto con una complejidad del tipo O(k*N) donde N es el número de tuplas y k el número de elementos más cercanos. Este tipo de consultas suele ser utilizada en muchos campos, como lo son minería de datos usado para estimar una función de densidad \textbf{F(x/$C_j$)} donde predice el valor $x$ para la clase $C_j$. Este algoritmo además es muy utilizado en el reconocimiento de patrones, por ejemplo imaginemos que dado el caso de una base de datos con rostros, se desea identificar los k rostros más similares a un rostro consulta. En este caso con las imágenes de los rostros se crean descriptores de la imágenes tanto para la base de datos como para la consulta en cuestión; este descriptor está compuesto por un vector lineal, el cual con el cálculo de la distancia entre dos vectores (descriptor) se obtiene la similitud entre ambos. Se debe considerar ciertos aspectos para establecer una variable K, dado que éste depende de los datos, de modo que si se selecciona un valor muy pequeño estos pueden ser ruidos (ser elementos de otra clase) y si se selecciona un K muy grande puede pasar exactamente lo mismo (abarcar elementos de otra clase). De tal manera que la elección del K debe ser elegida de acuerdo a la cantidad de datos.\\

Este algoritmo puede llevarse a un sinfín de otros usos, asociados a la clasificación y reconocimiento de patrones. En este punto es importante añadir que existen diferentes formas de emplear y aplicar el algoritmo de modo que en algunas situaciones es utilizado como un clasificador previo a análisis más potentes.   

\section{Objetivos}

El objetivo general de esta tesis es la creación de un software que abarca la selección y desarrollo de algoritmos que resuelven consultas kNN (k-nearest neighbors).  Dichos algoritmos serán seleccionados a partir de la literatura científica actual, sobre distintas plataformas paralelas.
\\
A continuación se mencionan los objetivos específicos que se han propuesto:

\begin{itemize}

\item Realizar un estudio del estado del arte acerca de la solución de consultas kNN sobre distintas
plataformas paralelas con sistema de memoria compartido: (1) servidor multi-núcleo, (2) Xeon Phi, y (3) GPU.

\item A partir de los algoritmos de la etapa anterior, seleccionar aquellos de mejor rendimiento, e
implementarlos.

\item Diseñar e implementar una aplicación con una interfaz orientada al usuario, que permita el manejo
de los diferentes algoritmos implementados en la etapa anterior. Además, que permita el uso de bases
de datos de distinta naturaleza y tamaño.

	
\end{itemize}



