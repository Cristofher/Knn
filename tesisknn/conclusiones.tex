\chapter[Conclusiones y Trabajos Futuros]{\label{ch:intro}Conclusión}


\section{Conclusiones}

En este trabajo se han buscado e implementados algoritmos presentes en la literatura para el procesamiento de consultas kNN en un software diseñado en usuarios que no tengan conocimientos previos en programación. Este software cuenta con algoritmos para procesar consultas kNN sobre tres tipos diferentes de plataformas paralelas: procesador multi-núcleo, coprocesador Xeon Phi y coprocesador GPU. 

Los algoritmos seleccionados luego de hacer una revisión bibliográfica, son aquellos algoritmos que presentan un rendimiento muy eficiente para las distintas plataformas involucradas, éstos se basan en la utilización de heaps y búsquedas exhaustivas, estos presentan buenos resultados en relación al tiempo. Estos fueron implementados exitosamente en un software que integra arquitecturas muy distintas, tales como son Xeon Phi (Intel) y GPU (NVIDIA). Cabe destacar que el software no afecta el funcionamiento del algoritmo en términos de tiempo, debido a la capacidad de JAVA de ejecutar la shell (Consola) a través de una rutina simple, de este modo es posible ejecutar los algoritmos programados en C, los cuales son compilados previamente en la instalación y luego solo son llamados para ser ejecutados.
\\
\\
Si no se cuenta con alguno de éstos dos coprocesadores el usuario no está forzado a adquirir uno de éstos, puede realizar las consultas desde su propio equipo utilizando tanto el modo secuencial como paralelo pero utilizando las características propias del equipo.


\section{Trabajos futuros}

El presente trabajo es la primera versión de este software, el cual en un futuro puede contar con mejoras, tales como:

\begin{itemize}
\item Llevar el software a otro sistema operativo cómo Windows, debido a ser el sistema operativo más utilizado. Además de intentar llevarlo a OSx, de manera que el software sea multi-plataforma y multi-copresadores. De esta manera se generaría un software más potente.
\item Intercomunicación entre el paralelismo multi-núcleo con el paralelismo en Xeon phi. Este trabajo futuro puede ser implementado de variados modos, de manera que se debe investigar y realizar la implementación de esta mejora. 
\item Intercomunicación entre el paralelismo multi-núcleo con el paralelismo en GPU. Este trabajo futuro puede ser implementado de variados modos, de manera que se debe investigar y realizar la implementación de esta mejora.
\item Añadir nuevos métodos de exportación a otros tipos de archivo que sean requeridos según ciertos criterios.
\item Añadir aparte de los 4 métodos principales, nuevos métodos los cuales utilicen otras propuestas para el algoritmo.
\item No solo se podrían añadir algoritmos kNN sino también otros asociados a métodos de clasificación, o bien el software puede contener una recopilación de algoritmos en diversas áreas.   

\end{itemize}
