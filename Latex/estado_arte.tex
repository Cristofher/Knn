\chapter[Estado del Arte]{\label{ch:estado-arte}Estado del Arte}

El paralelimos en computación es una técnica de programación la cual permite que varias intrucciones se puedan ejecutar similtaneamente. Se base en el principio de la división de grandes problemas en varios problemas mas pequeños, que son resueltos de forma concurrente.
\\
\section{\label{sec:des-intro}Tipos de paralelismo}
Existen diversos tipos de niveles de paralelismo que se describiran a continuación:
\subsection{Nivel de bit}
En la decada de 1970 hasta alrededor de 1986, la aceleración en la arquitectura de computadores lo logro aumentar el tamaño de una palabra, reduciendo asi el numero de instrucciones que un procesador debia realizar, de este modo se logro por ejemplo reducir el proceso de instrucciones en un procesador de 8 bits al sumar dos números de 16 bits.

\subsection{Nivel de instrucción}
Los avances en esta área se realizarón alrededor de 1980 y 1990, donde esto constaba con el reordenamiento y combinación de instrucciones en grupos que luego son ejecutadas en paralelo sin cambiar el resultado del programa. Actualmente los procesadores cuentan con "pipeline" de instrucciones de varias etapas, Cada etapa en el pipeline corresponde a una acción diferente que el procesador realiza en la instrucción correspondiente a la etapa; un procesador con un pipeline de N etapas puede tener hasta n instrucciones diferentes en diferentes etapas de finalización.
\subsection{Datos}
\subsection{Tareas}