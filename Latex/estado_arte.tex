\chapter[Estado del Arte]{\label{ch:estado-arte}Estado del Arte}

El paralelimos en computación es una técnica de programación la cual permite que varias intrucciones se puedan ejecutar similtaneamente. Se base en el principio de la división de grandes problemas en varios problemas mas pequeños, que son resueltos de forma concurrente.
\\
\section{Tipos de paralelismo}
Existen diversos tipos de niveles de paralelismo que se describiran a continuación:
\subsection{Nivel de bit}
En la decada de 1970 hasta alrededor de 1986, la aceleración en la arquitectura de computadores lo logro aumentar el tamaño de una palabra, reduciendo asi el numero de instrucciones que un procesador debia realizar, de este modo se logro por ejemplo reducir el proceso de instrucciones en un procesador de 8 bits al sumar dos números de 16 bits.

\subsection{Nivel de instrucción}
Los avances en esta área se realizarón alrededor de 1980 y 1990, donde esto constaba con el reordenamiento y combinación de instrucciones en grupos que luego son ejecutadas en paralelo sin cambiar el resultado del programa. Actualmente los procesadores cuentan con "pipeline" de instrucciones de varias etapas, Cada etapa en el pipeline corresponde a una acción diferente que el procesador realiza en la instrucción correspondiente a la etapa; un procesador con un pipeline de N etapas puede tener hasta n instrucciones diferentes en diferentes etapas de finalización
\subsection{Datos}

El paralelismo de datos es el paralelismo inherente en programas con ciclos, de este modo corresponde a la subdivición de los datos de entrada a un programa, de modo que a cada procesador le corresponda un subconjunto de los datos divididos.\\
Este paradigma es mejor orientado para operaciones con matrices o vectores, debido que se realiza la misma operación sobra cada uno de los elementos.

\subsection{Tareas}

El paralelismo de tareas es la característica de un programa paralelo en la que cálculos completamente diferentes se pueden realizar en cualquier conjunto igual o diferente de datos. Esto contrasta con el paralelismo de datos, donde se realiza el mismo cálculo en distintos o mismos grupos de datos. El paralelismo de tareas por lo general no escala con el tamaño de un problema.
\\

Actualmente la computación paralela ha aumentado su interes debido a las restricciones fisicas, en base a la arquitectura computacional que se han masificados los computadores con procesadores múltinúcleos. Según el hardware que posee el equipo se puede clasificar el nivel de paralelismo que admite, como por ejemplo los computadores multinucleos y multiprocesos permiten realizar varios procesamientos en una misma maquina, mientras que a su vez los clusters, los MPP y los grids emplean varios computadores para trabajar en la misma tarea.
\\
En los ultimos años el uso de coprocesadores ha sido un área de basto interes en temas de investigación en el campo de la computación paralela, con el fin de acelerar procesos secuenciales, los cuales pueden ser acelerados con los procesadores propios del equipo o bien con coprocesadores integrados.

\section{Procesadores y Coprocesadores} 

\subsection{Procesadores}

Un multiprocesador es un conjunto de dos o más procesadores los cuales se encuentran a menudo integrados en un solo circuito. Hoy en día los multiprocesadores mas conocidos son dual-core, quad-core, octa-core con dos, cuatro y ocho núcleos respectivamente. Estos multiprocesadores permiten un paralelismo a nivel de hilos (threads) conocido como TLP (thread-level parallelism), aca no se consideran los microprocesadores en paquetes separados.
\\
Es posible que se realice un procesamiento simultaneo utilizando dos o más procesadores en un pc, o bien realizando la conección de dos o más pc, este ultimo suele ser mas restrintivo debido a las caracteristicas de cada uno de los pcs conectados, que en algunos casos se realiza la superposición entre ellos.

\subsection{Coprocesadores}

Un coprocesador es un microprocesador de un ordenador utilizado como suplemento de las funciones del procesador principal (la CPU). Las operaciones ejecutadas por uno de estos coprocesadores pueden ser operaciones de aritmética, procesamiento gráfico, de señales, procesado de texto o criptografía, etcétera.
\\
Las lineas más famosas de coprocesadores en el ámbito de la investigación son INTEL con su tecnología MIC y NVIDIA GPU con su tecnología CUDA

\subsection{Intel}
Intel a desarrollado una tecnología propía, orientada a computo intensivo en computación paralela conocido como MIC (Many integrated core) asociados a su familia Xeon, similar a la Nvidia Tesla que forman una arquitectura de alto rendimiento. \\
Se puede considerar que cada núcleo del co-procesador Xeon Phi es una unidad de ejecución
multi-hilo completamente funcional. Poseen un rendimiento de hasta 1,2 teraFlops  (operaciones de coma flotante por segundo) por coprocesador para algunas de las aplicaciones más exigentes de hoy día.

\subsection{NVIDIA}
NVIDIA se ha dedicado a la elaboración de GPUs (Unidades de procesamiento gráfico) los cuales son procesadores de varios núcleos que ofrecen un alto rendimiento, tambien dedicados al procesamiento de graficos o bien operaciones de punto flotante. \\
NVIDIA ha desarrollado una tecnología de acuerdo a su hardware conocida como CUDA, que cuenta con una plataforma de computación paralela y un modelo de programación. Esta tecnología permite aumentos considerables en el rendimiento computacional al aprovechar la potencia de la GPU.

\section{KNN (k-nearest neighbors)}

Conocido por sus siglas en ingles como KNN (K-nearest neighbors) k vecinos más cercanos, es un método de clasificación de objetos, siendo la clafisicación de objetos un área muy importante en variados campos de investigación y de aplicaciones, como lo son el reconocimiento de patrones, inteligencia artificial, psicología cognitiva, analísis de videos y medicina.  