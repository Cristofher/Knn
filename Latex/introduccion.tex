
\chapter[Introducción]{\label{ch:intro}Introducción}

texto...


\section{Objetivos}

El objetivo general de esta tesis se centra en la recopilación de algoritmos de vecinos más cercanos (k-nearest neighbor) encontrados en la literatura que proponene una solución más optima en tiempos de ejecución, para esto se establecio que los objetivos que cumplen con esta condición son aquellos que han sido propuestos en las lineas de paralelismo, donde a su vez se emplean heap para la resolución de este algoritmo.

Para poder realizar la unión de los diferentes algoritmos paralelos, se propone un software el cual permite la agrupación de algoritmos de diferentes plataformas paralelas, además de añadir el algoritmo secuencial base como un agregado. En base a lo anteriormente mensionado se pretende que el software sea diseñado en consideración a usuarios los cuales no esten familiarizados con la programación de estos metodos, de modo que la utilización del software no se vea ligada a sus conocimientos en áreas de programación, además tomar en consideración que la 


Los objetivos específicos son los siguientes:

\begin{itemize}
\item Realizar la interfaz gráfica del software
	Esta se realiza en JAVA debido a su capacidad de ser interpretada tanto en Windows, Linux y OSx
\item objetivo-especifico-2
\item objetivo-especifico-N
\end{itemize}



